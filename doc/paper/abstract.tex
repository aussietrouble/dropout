\begin{abstract}

  We propose a procedure to iteratively zero out entries in the
  data matrix for linear regression, drawing inspiration from a recent
  heuristic for training large multi-layer neural networks for image
  processing. Rather than using dropout as a form of regularization,
  our rationale for zeroing out data is to enable more
  efficient computation of a linear model, while controlling the
  excess risk that this data corruption incurs.  To achieve this we
  make use of recent algorithmic advances in ``subspace embedding''
  techniques for low rank approximation and least squares estimation
  from sparse data matrices.  We develop these ideas in the setting of
  large scale linear regression, focusing on the regime where the
  sample size is large relative to the number of variables.  Our
  analysis and simulations show that the dropout method has a
  favorable risk-computation tradeoff compared with simple subsampling
  of the data.  We also apply these ideas to the setting where many of
  the variables are irrelevant.  For this case we propose a dropout
  version of the alternating direction method of multipliers 
  algorithm for approximate $\ell_1$-regularized least squares.
\end{abstract}

